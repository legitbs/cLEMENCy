\documentclass{book}

\usepackage{fontspec}
\usepackage{xunicode}
%% \setmainfont{Ubuntu}
%% \setmonofont{Ubuntu Mono}


\begin{document}
%% \tableofcontents
\chapter{Instruction Set}
Unless specified otherwise, all math is unsigned for integer arithmetic. All
floating point math is signed. All \texttt{r\textit{X}} values can reference
a general purpose
register from 0 to 31. Anytime the format
\texttt{r\textit{X}:r\textit{X}+\textit{Y}}
is seen, the instruction will work on registers
\texttt{r\textit{X}} through and including \texttt{r\textit{X}+\textit{Y}} based
on the value of \textit{Y}. The \texttt{UF}
field controls if the flags get updated for the instruction.


\section{AD | Add}
\begin{tabular}{lcr*{4}{|lcr}|lcr}
0 & & 6 &
7 & & 11 &
12 & & 16 &
17 & & 21 &
22 & & 25 &
& 26 & \\
\hline
\multicolumn{3}{c|}{0000000} &
\multicolumn{3}{c|}{rA} &
\multicolumn{3}{c|}{rB} &
\multicolumn{3}{c|}{rC} &
\multicolumn{3}{c|}{0000} &
\multicolumn{3}{c}{UF} \\

\end{tabular}

\begin{description}
\item [Format:] \texttt{AD rA, rB, rC}
\item [Purpose:] To add two 27-bit integer registers together
\item [Description:] The 27-bit value in GPR \texttt{rC} is added to the 27-bit
  value in GPR \texttt{rB}, the result is placed in GPR \texttt{rA}
\item [Operation:] \(\mathtt{rA} \leftarrow \mathtt{rB} + \mathtt{rC}\)
\item [Flags affected:] \texttt{Z C O S}
\end{description}
\end{document}
